%%%%%%%%%%%%%%%%%%%%%%%%%%%%%%%%%%%%%
%%   Folha de aprova��o
%%%%%%%%%%%%%%%%%%%%%%%%%%%%%%%%%%%%%


\thispagestyle{empty}

\begin{spacing}{1.0}

\begin{center}
\end{center}

\vspace {1.0 cm}

\begin{center}
\begin{Large}
{\bf \uppercase{Aspectos de Vis�o Computacional no desenvolvimento do \textit{toolkit} Horus}
} \\[0.2 cm]
\end{Large}
\end{center}

\vspace {1.0 cm}

\begin{flushright}
\begin{minipage}[t]{8.5 cm}
Monografia apresentada � Universidade C�ndido Mendes como requisito obrigat�rio para a obten��o do grau de Bacharel em Ci�ncias da Computa��o.
\end{minipage}
\end{flushright}

\vspace {1.5 cm}

\begin{center}
\begin{large}
Aprovada em \_\_\_\_ de \_\_\_\_\_\_\_\_\_\_\_\_ de 2009. \\[0.2 cm]
\end{large}
\end{center}

\vspace {1.5 cm}

\begin{center}
\begin{large}
BANCA EXAMINADORA \\[0.2 cm]
\end{large}
\end{center}

\vspace {1 cm}

\begin{center}
\line(1,0){300}\\ 
Prof. �talo de Oliveira Matias - Orientador \\
D.Sc.em Sistemas Computacionais pela UFRJ\\

\end{center}

\vspace {1.0 cm}

\begin{center}
\line(1,0){300}\\ 
Prof. Dalessandro Soares Vianna\\
D.Sc. em Inform�tica pela PUC-Rio
\end{center}

\vspace {1.0 cm}

\begin{center}
\line(1,0){300}\\ 
Prof. Ferm�n Alfredo Tang Montan�\\
D.Sc. em Engenharia de Produ��o pela UFRJ
\end{center}

\end{spacing}
\newpage
