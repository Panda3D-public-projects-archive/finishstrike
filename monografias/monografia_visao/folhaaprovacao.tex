%%%%%%%%%%%%%%%%%%%%%%%%%%%%%%%%%%%%%
%%   Folha de aprova��o
%%%%%%%%%%%%%%%%%%%%%%%%%%%%%%%%%%%%%


\thispagestyle{empty}

\begin{spacing}{1.0}

\begin{center}
\begin{large}
\uppercase{
}
\end{large}
\end{center}

\vspace {1.0 cm}

\begin{center}
\begin{large}
{\bf \uppercase{Aspectos de vis�o computacional no \textit{Toolkit} Horus}
} \\[0.2 cm]
\end{large}
\end{center}

\vspace {1.0 cm}

\begin{flushright}
\begin{minipage}[t]{8.5 cm}
Monografia apresentada � Universidade C�ndido Mendes como requisito obrigat�rio para a obten��o do grau de Bacharel em Cin�ncias da Computa��o.
\end{minipage}
\end{flushright}

\vspace {1.5 cm}

\begin{center}
\begin{large}
Aprovada em \_\_\_\_ de \_\_\_\_\_\_\_\_\_\_\_\_ de 2009. \\[0.2 cm]
\end{large}
\end{center}

\vspace {1.5 cm}

\begin{center}
\begin{large}
BANCA EXAMINADORA \\[0.2 cm]
\end{large}
\end{center}

\vspace {1.5 cm}

\begin{center}
\line(1,0){300}\\ 
Prof. D.Sc. �talo de Oliveira Matias - Orientador \\
Doutor pela UFRJ
\end{center}

\vspace {1.0 cm}

\begin{center}
\line(1,0){300}\\ 
Prof. D.Sc. Dalessandro Soares Vianna\\
Doutor pela PUC-Rio
\end{center}

\vspace {1.0 cm}

\begin{center}
\line(1,0){300}\\ 
Prof. D.Sc. Ferm�n Alfredo Tang Montan�\\
Doutor pela UFRJ
\end{center}


\end{spacing}
\newpage
