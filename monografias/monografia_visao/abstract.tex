%%%%%%%%%%%%%%%%%%%%%%%%%%%%%%%%%%%%%
%%   Abstract
%%%%%%%%%%%%%%%%%%%%%%%%%%%%%%%%%%%%%


\chapter*{Abstract}
This work presents the concepts and algorithms involved in the construction of an Open Source toolkit, named Horus, used in development and control of applications involving intelligent agents, focusing on two central problems. The first problem relates to the handling of autonomous intelligent agent in unknown environments. The second problem relates to computer vision, where the agent should be able to extract information from the environment through the use of virtual or real cameras. In addition to algorithms for computer vision and autonomous movement, the Horus also provides abstractions for the development and configuration of intelligent agents with their main components and devices. While Horus was also developed with a focus on movement of autonomous intelligent agents, this work is focused on the computer vision algorithms and abstractions for creation and configuration of intelligent agents. As a proof of concept and validation of the functionality provided by the toolkit Horus, were also developed three distinct applications: PyANPR, Teseu and Ariadnes. PyANPR is an automatic number plate recognition web system. Teseu is a system that simulates the automatic mapping of an unknown environment by an intelligent agent. Ariadnes is a system that simulates the movement of an autonomous intelligent agent in an unfamiliar environment using techniques of computer vision and unknown environments mapping. This work will also detail the PyANPR aplication and the computer vision aspects of the Ariadnes simulator.


\vspace{0.5cm}

Keywords: Computer Vision, Environment Mapping, Horus

