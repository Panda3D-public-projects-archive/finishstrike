%%%%%%%%%%%%%%%%%%%%%%%%%%%%%%%%%%%%%
%%   CONCLUS�ES E TRABALHOS FUTUROS
%%%%%%%%%%%%%%%%%%%%%%%%%%%%%%%%%%%%%


\chapter{Conclus�es e Trabalhos Futuros}
\label{cap:conclusao}

Neste trabalho foi apresentada uma abordagem do c�lculo autom�tico do fator de cubica��o de pilhas de madeira, baseado em fotos. Para encontrar esse fator, utilizamos algumas das t�cnicas existentes para o c�lculo de um limiar e aplica��o do algoritmo \textit{threshold}. As t�cnicas utilizadas foram: o limiar por m�dia, limiar adaptativo, segmenta��o por Entropia de Shannon e segmenta��o por Entropia de Tsallis, conforme exposto no Cap�tulo \ref{cap:metodo_proposto}. O processo de segmenta��o � feito extraindo-se da foto principal da tora uma amostra, e calculando o fator sobre ela.

Com base nos experimentos, constatamos que a estrat�gia de subdividir a imagem e calcular um limiar independente para cada janela contribui consideravelmente para a melhora dos resultados. Em rela��o � sua qualidade nos algoritmos, esta varia de acordo com o tipo de foto. Em especial, para fotos com fortes contrastes de ilumina��o, o que � bem claro nas fotos das pilhas 4 e 5 (Cap�tulo \ref{cap:resultados_obtidos}, Figura \ref{fig:amostras_pilhas}), todos os algoritmos tenderam a ter uma queda na performance.

Contudo, destacamos o desempenho do limiar adaptativo, que obteve um dos melhores resultados, com consider�vel precis�o. Quanto aos resultados obtidos com a segmenta��o por entropia de Tsallis observou-se que, pela possibilidade de variar o valor do $q$, o algoritmo � um dos que possuem mais chances de acertar um limiar �timo mais adequado para a resolu��o do problema, se forem levados em conta outros fatores para a escolha deste. Nas imagens com menores contrastes na ilumina��o obtivemos melhores resultados com $q=0,5$, e para as fotos com maiores contrastes, tivemos um melhor desempenho com $q=-1$, $q=-1,5$ e $q=-2$, o que pode sugerir um certo n�vel de superextensividade do sistema.

Uma poss�vel extens�o do presente trabalho seria desenvolver uma forma de escolher automaticamente uma metodologia ou, no caso da Entropia de Tsallis, escolher automaticamente um valor de $q$ mais adequado para cada imagem, individualmente. Outro trabalho que pode ser desenvolvido a partir desse � uma forma de, tomando-se uma foto completa, identificar e selecionar automaticamente a regi�o a ser calculada.
