%%%%%%%%%%%%%%%%%%%%%%%%%%%%%%%%%%%%%
%%   Resumo
%%%%%%%%%%%%%%%%%%%%%%%%%%%%%%%%%%%%%


\chapter*{Resumo}

Este trabalho foi realizado com o intuito de comparar e avaliar quatro das principais t�cnicas de limiariza��o e segmenta��o de imagens, aplicadas ao processamento de imagens de toras de madeira, como solu��o para o c�lculo de fatores de empilhamento. Entre essas t�cnicas, propusemos a segmenta��o realizada atrav�s da utiliza��o da Entropia N�o Extensiva, tamb�m conhecida como Entropia de Tsallis \cite{Tsallis_1988,Tsallis_1999,Tsallis_2001}, aplicada a este problema. Para isso, foram implementadas cada uma das t�cnicas e testadas atrav�s de um software que foi desenvolvido. Foram realizados testes com fotos de pilhas de madeira empilhadas manualmente sobre o solo; e foram comparados os resultados de cada t�cnica com as medi��es manuais realizadas em campo. O objetivo deste trabalho �, atrav�s dos resultados dos experimentos realizados, verificar qual dos algoritmos testados se adequa melhor ao c�lculo autom�tico do fator de cubica��o, utilizando segmenta��o de imagens. 

\vspace{0.5cm}
Palavras-chave: Fatores de empilhamento, fotografias digitais, Entropia de Tsallis, Segmenta��o de imagens
