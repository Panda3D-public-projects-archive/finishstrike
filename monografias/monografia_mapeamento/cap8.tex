%%%%%%%%%%%%%%%%%%%%%%%%%%%%%%%%%%%%%
%%   Aplica��o do ambiente virtual com um rob� aut�nomo
%%%%%%%%%%%%%%%%%%%%%%%%%%%%%%%%%%%%%


\chapter{Aplica��o do ambiente virtual com um rob� aut�nomo}
\label{cap:aplicacao_autonomo}
A fim de produzir uma prova de conceito, foi implementada uma aplica��o onde foi poss�vel demonstrar a efici�ncia do algoritmo de mapeamento. Foi produzido um simulador com gravidade, colis�o, rederiza��o de texturas, objetos e atores (o agente nesse caso) e, tendo esse simulador como base, foi criada a aplica��o que tem um agente utilizando-se do Toolkit H�rus para mapeamento e navega��o dentro do ambiente simulado.
\section{Simulador}
O simulador desenvolvido foi tratado como uma aplica��o a parte, pois n�o estava inclu�do no escopo do projeto, por�m tendo em vista a necessidade de um simulador customizado para a realidade de um agente m�vel (e na linguagem selecionada) o simulador foi incluido ao projeto. 

\subsection{Requisitos para o ambiente virtual}
Foram utilizados alguns softwares para que fosse poss�vel a cria��o e renderiza��o do ambiente. Esses softwares foram: Blender e Panda3d.
Como uma das linguagens utilizadas foi Python, buscou-se ferramentas que fossem compat�veis (na verdade ambas s�o em Python) com as linguagens utilizadas.

\subsubsection{Blender}
O Blender � um produto da Blender Fundation, que � open source e desenvolvido em Python para modelagem de objetos 3D que est� dispon�vel para v�rios sistemas operacionais sobre a licen�a GNU (General Public License).

\subsubsection{Panda3D}
O Panda3D � um produto da equipe de desenvolvimento da Walt Disney para renderiza��o de jogos e ambientes virtuais em terceira dimens�o. Est�o sobre licen�a da BSD License (com algumas modifica��es para adequa��o a realidade do projeto).

\subsection{Ambiente}
