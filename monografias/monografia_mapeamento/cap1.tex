%%%%%%%%%%%%%%%%%%%%%%%%%%%%%%%%%%%%%
%%  INTELIG�NCIA ARTIFICIAL
%%%%%%%%%%%%%%%%%%%%%%%%%%%%%%%%%%%%%


\chapter{Intelig�ncia Artificial}
\label{cap:inteligencia_artificial}
\section{Agentes Inteligentes} 
Um Agente, por defini��o, e todo elemento ou entidade autonoma que pode perceber seu ambiente por algum meio cognitivo ou sensorial e de agir sober esse ambiente por interm�dio de atuadores.
Algumas defini��es do termo agente na lingua portuguesa tais como "O que opera ou � capaz de operar", "O que promove neg�cios alheios" e "Autor".
Existem defini��es de agentes em v�rias �reas do conhecimento
humano.

\begin{itemize} 
	\item Sociologia 
	\item Economia 
	\item Comportamento Animal
	\item Software
	\item Rob�tica 
\end{itemize}


\section{Teste de Turing}
\section{T�cnicas de IA}
\subsection{RNA}
\subsection{Algoritmos Gen�ticos}
\subsection{L�gica Fuzzy}
