%%%%%%%%%%%%%%%%%%%%%%%%%%%%%%%%%%%%%
%%   Abstract
%%%%%%%%%%%%%%%%%%%%%%%%%%%%%%%%%%%%%


\chapter*{Abstract}

\vspace{0.5cm}
This work demonstrates the use of the Horus Toolkit, with focus on mapping module, for creating intelligent agents for unknown environments where act recognizing and exploring the environment. The philosophy of mapping SLAM (Simultaneous Location and Mapping) was used and other techniques developed to solve the problems of the proposed task-environments.

Among the other techniques is the presentation of a mapping method based on marks (markpoints) that he creates during its navigation in order to identify further sites where already been.

Also present a method of mapping based on the idea of punishment and reward, in which the agent gets credit for finding new markers and lose them if find twice the same marks, thus creating a condition to stop your loss when the desired answer to a previously established requirement .

In this work, the agent uses only two types of sensors to accomplish its task of navigation, simultaneous localization and mapping: laser and odometer.

Finally, the work aims to create a proof of concept for the module developed thus showing its operation. The SLAM module developed and responsible only for the tasks of simultaneous localization and mapping, the application uses the agent responsible for the grant, and its navigation devices used.

Keywords: MAPPING, LOCALIZATION, SIMULTANEOUS LOCALIZATION AND MAPPING, SLAM, TOOLKIT, H�RUS, INTELIGENT AGENT.  

