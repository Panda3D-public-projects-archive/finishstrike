%%%%%%%%%%%%%%%%%%%%%%%%%%%%%%%%%%%%%
%%  ROB�TICA
%%%%%%%%%%%%%%%%%%%%%%%%%%%%%%%%%%%%%


\chapter{Rob�tica}
\label{cap:robotica}
A ci�ncia rob�tica � repons�vel pela parte da tecnologia que tem por intuito otimizar tarefas feitas por humanos e, em alguns casos, substitu�-los por motivos que v�o desde a preserva��o da integridade do ser humano at� mesmo a ocupa��o de seu cargo de trabalho. Alheios a um mundo de filmes e preconceitos, os rob�s tornam os resultados dos servi�os melhores e sua precis�o � muito mais que a de um funcion�rio.

\section{Dispositivos}

A rob�tica se utiliza de v�rios dispositivos para emular os sentidos e as rea��es humanas em determinadas situa��es. Esses dispositivos tratam diferentemente cada um dos sentidos humanos.
Algumas categorias de dispositivos s�o:
\begin{enumerate} 
\item Sensores: Dispositivos diversos que "sentem" o mundo e convertem isso para dados pass�ves de serem analisados.
\item Efetuadores
\item Percep��o
\item Localiza��o
\item Mapeamento
\item Planejamento
\item Movimento: S�o as t�cnicas que podem ser utilizadas para 
\item Controle:
\end{enumerate}

\section{Rob�tica M�vel}
