%%%%%%%%%%%%%%%%%%%%%%%%%%%%%%%%%%%%%
%%   Introdu��o
%%%%%%%%%%%%%%%%%%%%%%%%%%%%%%%%%%%%%


\chapter{Introdu��o}
A rob�tica era um sonho at� pouco tempo atr�s, hoje em dia a sua exist�ncia � t�o comum que muitas vezes nos passam desapercebidos grandes avan�os da �rea. 
Cada vez mais robos, ou melhor, agentes inteligentes est�o presentes em nosso dia-a-dia a fim de facilitar nossos afazerese com isso nos tornar mais produtivos. 

Com esse intuito de otimizar nosso tempo, a rob�tica voltou-se ainda mais para intelig�ncia artificial de seus agentes tornando-os mais independentes e mais capazes de solucionar problemas. Assim os agentes poderiam solucionar problemas previamente definidos ou n�o, este �ltimo atrav�s de t�cnicas avan�adas de intelig�ncia artificial.

A rob�tica m�vel � um campo da rob�tica que estuda as vantagens da mobilidade dos agentes. Essa mobilidade os tornam capazes de avan�ar ainda mais em ambientes de dif�cil acesso ao ser humano, por exemplo, pesquisas subaqu�ticas ou, at� mesmo, em outros planetas. Alguns fatores seram apresentados nessa monografia levam em considera��o conceitos como: intelig�ncia artificial, agentes inteligentes, mapeamento, localiza��o e outros conceitos da rob�tica m�vel.

No cap�tulo \ref{cap:inteligencia_artificial} foram introduzidos conceitos importantes de inteligencia artificial, como estruturas de agentes, j� no cap�tulo \ref{cap:ambiente} abordou-se sobre o Ambiente tarefa e como ele influencia no projeto de um agente, o cap�tulo \ref{cap:robotica} foram apresentados t�picos sobre arquiteturas de rob�s relevantes para o desenvolvimento da aplica��o, no cap�tulo \ref{cap:SLAM} o foco maior foi no SLAM e como ele foi utilizado na aplicacao, enquanto no cap�tulo \ref{cap:o_toolkit_horus} foram demonstrados, do \textit{Toolkit} Horus, os modulos de visao, mapeamento, core e utils e o cap�tulo \ref{cap:aplicacao_autonomo} refere-se a aplica��o com um agente aut�nomo em um ambiente desconhecido.


