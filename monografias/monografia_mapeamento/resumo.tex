%%%%%%%%%%%%%%%%%%%%%%%%%%%%%%%%%%%%%
%%   Resumo
%%%%%%%%%%%%%%%%%%%%%%%%%%%%%%%%%%%%%


\chapter*{Resumo}

O presente trabalho demonstra a utiliza��o do Toolkit Horus, com foco no m�dulo de mapeamento, para a cria��o de agentes inteligentes para ambientes desconhecidos onde agiram reconhecendo e explorando o ambiente. Foi usada a filosofia de mapeamento SLAM (Simultaneos Location and Mapping), al�m de outras t�cnicas desenvolvidas para resolver os problemas dos ambientes-tarefa propostos.

Dentre as outras t�cnicas est� a apresenta��o de um m�todo de mapeamento baseado em marcos (markpoints) que o agente cria durante a sua navega��o a fim de identificar posteriormente lugares nos quais j� esteve.

Tamb�m apresentaremos um m�todo de mapeamento pr�prio baseado na id�ia de puni��o e recompensa, no qual o agente ganha ao encontrar novos marcos e perde ao reencontr�-los, criando assim uma condi��o de parada desejada quando a sua perda atender a um requisito previamente criado.

Neste trabalho, o agente utiliza apenas dois tipos de sensores para comprir sua tarefa de navega��o, localiza��o e mapeamento simult�neo: laser e od�metro.

Finalmente, o trabalho pretende criar uma prova de conceito para o m�dulo desenvolvido comprovando assim o seu funcionamento. O m�dulo SLAM desenvolvido � respons�vel apenas pelas tarefas de localiza��o e mapeamento simult�neo, sendo a aplica��o que o utiliza a respons�vel por conceder o agente, sua navega��o e os dispositivos usados.

\vspace{0.5cm}
Palavras-chave: MAPEAMENTO, LOCALIZA��O, SIMULTANEOUS LOCALIZATION AND MAPPING, SLAM, TOOLKIT, AGENTE INTELIGENTE.
