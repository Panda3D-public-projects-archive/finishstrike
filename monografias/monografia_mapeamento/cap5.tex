%%%%%%%%%%%%%%%%%%%%%%%%%%%%%%%%%%%%%
%%  O Tool Kit Horus
%%%%%%%%%%%%%%%%%%%%%%%%%%%%%%%%%%%%%


\chapter{O Toolkit Horus}
\label{cap:o_toolkit_horus}

O Hor�s � um toolkit, ou seja, uma cole��o de ferramentas, nesse caso m�dulos, que servem para gerenciar agentes inteligentes, escrito em Python. Duas partes est�o sendo desenvovdidas, a princ�pio, estas s�o: M�dulo de Vis�o e M�dulo de Mapeamento.

No m�dulo de vis�o � onde se localiza os mais v�riados algoritmos de vis�o computacional e no m�dulo de mapeamento trata-se do problema de mapear ambientes a partir de dispositivos de leitura do ambiente e de efetuadores.
\section{Objetivo}

O objetivo do toolkit � prover ferramentas necess�rias para produ��o de agentes inteligentes. 

\section{Arquitetura}

Demonstrar a arquitetura utilizada.

\section{M�dulos do Horus}

Principais funcionalidades de cada um dos m�dulos, de forma mais explicada. .

\begin{itemize}
\item Core do Horus: Cont�m m�dulos que ser�o utilizados como suporte para os m�dulos principais. Sendo assim n�o necessitam ser utilizados diretamente pelo usu�rio.
\item Modulo de Vis�o: Tem por objetivo tratar as principais t�cnicas de vis�o computacional de um agente inteligente.
\item Modulo de Mapeamento: O m�dulo de mapeamento � respons�vel por gerenciar os tipos mapeamento bem como os dispositivos utilizados para mapear e navegar no ambiente.
\end{itemize}

