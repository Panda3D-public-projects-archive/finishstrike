%%%%%%%%%%%%%%%%%%%%%%%%%%%%%%%%%%%%%
%%  AMBIENTE
%%%%%%%%%%%%%%%%%%%%%%%%%%%%%%%%%%%%%


\chapter{Ambiente}
\label{cap:ambiente}

De uma forma geral, um ambiente � um conjunto das condi��es, situa��es e/ou condi��es onde existe determinado objeto ou ocorre uma determinada a��o. No caso desse projeto o ambiente significa o local onde esta inserido um agente deve explorar, mapear e navegar. 

Antes de imaginar um agente inteligente, deve-se pensar em um ambiente de tarefas, que ser� utilizado para ser o "problema" em si que os agentes devem utilizar para gerar suas "solu��es". 

\section{Especificando um ambiente}

O ambiente de tarefa, onde o agente ir� realizar suas tarefas, tem que ser o mais completo poss�vel. Lembrado que o agente deve ser inserido em um ambiente que seja apropriado ao seu prop�sito. Alguns fatores ser�o levados em considera��o ao projetar um agente:
\begin{enumerate}
\item Medida de desempenho: Qual ser� a o objetivo do projeto? minimizar custos, minimizar consumo de energia, otimizar rotas, melhorar mapeamente etc.
\item Ambiente: Que tipo de ambiente? Est�tico, determin�stico, previs�vel etc.
\item Agente: Que tipo de agente? Human�ide, pr�prio para ambiente aqu�tico, voador etc.
\item Sensores: Que tipo de sensores devem ser inseridos no agente? Lasers, sonares, cam�ras etc.
\end{enumerate}

Esses fatores ser�o de enorme import�ncia ao se determinar um ambiente.

\section{Propriedades de ambientes de tarefas}

A quantidade de ambientes de tarefas que podem ser criados � enorme, no entanto, podemos subdividir em categorias. Essas categorias determinam um projeto apropriado para o agente e as principais t�cnicas que devem ser implementadas no agente. As categorias s�o:

\subsection{Completamente observ�vel e parcialmente observ�vel}

Se os sensores de um agente permitem acesso ao estado completo do ambiente em cada instante, dize-se que o ambiente � completamente observ�vel. Um ambiente tarefa � completamente observ�vel se os sensores detectam todos os aspectos que s�o relevantes para escolha de uma a��o, relev�ncia tal que depende da medida de desempenho. Um ambiente � parcialmente observ�vel quando devido a ru�do, sensores impresisos, sensores mal configurados ou parte do ambiente est�o ausentes nos dados do sensor.
