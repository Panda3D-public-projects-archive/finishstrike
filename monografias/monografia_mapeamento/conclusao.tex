%%%%%%%%%%%%%%%%%%%%%%%%%%%%%%%%%%%%%
%%   CONCLUS�ES E TRABALHOS FUTUROS
%%%%%%%%%%%%%%%%%%%%%%%%%%%%%%%%%%%%%


\chapter{Conclus�es e Trabalhos Futuros}
\label{cap:conclusao}

H� muito que melhorar para garantir que o processo de mapeamento seja mais produtivo.
Como trabalhos futuros iremos em primeiro momento concluir o ransac, m�todo respons�vel por encontrar landmarks a partir de pontos extra�dos com leitura de lasers. Com isso, o agente ser� capaz de identificar paredes e poder� us�-las para a sua navega��o e mapeamento.

Apesar de o m�todo ransac estar pronto e fazer parte do nosso m�dulo SLAM, ele ainda n�o est� em uso, pois n�o foi necess�rio no ambiente-tarefa das aplica��es. Contudo, testes individuais executados no m�todo indicam um bom funcionamento com taxas de erros muito baixa. Entretanto na aplica��o o numero de landmarks encontrados esta abaixo do esperado. Os ajustes no ransac ser�o o de desconsiderar as leituras dos lasers das pontas, por possu�rem uma discrep�ncia em rela��o a outras leitoras e gerarem erro no ransac, excluir os pontos com o valor considerado infinito na aplica��o e fazer o calculo do intervalo de confian�a da linha encontrada.

Com o ransac pronto iniciaremos o modulo que criara um grafo completo a partir dos marcos encontrados e com os landmarks � poss�vel fazer a poda do grafo. Com esse grafo ajustado constru�do, buscaremos o melhor caminho com algoritmo gen�tico.
No atual ambiente tarefa n�o foi considerada os erros dos sensores. Com a adi��o desses erros, m�dulos de tratamento de incerteza utilizando L�gica Fuzzy e Filtro de Kalman passaram entrega o Horus.
